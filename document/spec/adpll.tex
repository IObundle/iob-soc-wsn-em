\section{ADPLL}
\label{sec:adpll}

In this section the detailed specifications of the All-Digital Phase-Locked-loop
(ADPLL) and its internal mixed-signal blocks are presented.


\subsection{General specifications}

The general specifications of the ADPLL block are given in
Table~\ref{tab:adpll_specs}.

\begin{table}[!h]
  \centering
  \begin{tabular}{|p{5cm}|p{2cm}|p{7cm}|}
    \hline
    \rowcolor{tab-green}
    & {\bf Spec} & {\bf Notes} \\
    \hline \hline {\bf Ouput Frequency} & X - X MHz &  \\
    \hline {\bf Reference Frequency} & 32 MHz &   \\
    \hline {\bf RX RMS Integrated Jitter} & X ps & phase-noise integration \\
    \hline {\bf RX Integration Range} & 0.1-100 MHz & \\
    \hline {\bf RX Loop Bandwidth} &  X MHz & \\
    \hline {\bf TX Loop Bandwidth} & X MHz & e \\
    \hline {\bf TX FSK error} & X $\%$ &  \\
    \hline {\bf Power Consumption} &  X mW & 1.2 V supply\\
    \hline {\bf Area} & X mm$^2$ &  \\
    \hline                                   
  \end{tabular}
  \caption{General specfications.}
  \label{tab:adpll_specs}
\end{table}



\subsection{Block diagram}

The ADPLL is comprised of a digital core unit, a digital-controlled
oscillator (DCO), a time-to-digital converter (TDC), and a clock
retiming unit, as can be seen in Fig. \ref{fig:adpll_bd}.

Details about the TDC and the DCO blocks can be found in section \ref{sec:tdc} and \ref{sec:dco}, respectively.

The retiming unit synchronizes the DCO and adpll\_ctr clock domains.
This is achieved by oversampling of the FREF clock by the high-rate
DCO clock. To reduce metastability issues, the retiming unit employs
custom-designed flip-flops.




\begin{figure}[!h]
  \centering
      {\includegraphics[width=12cm]{./drawings/adpll_bd.png}}
  \caption{Diagram block}
  \label{fig:adpll_bd}
\end{figure}



\subsection{Interface signals}

The ADPLL interface signals are shown in Table~\ref{tab:adpll_interface}. 

\begin{table}[!h]
  \centering
    \begin{tabular}{|p{3cm}|p{1.5cm}|p{2cm}|p{1.5cm}|p{7cm}|}
      \hline
      \rowcolor{tab-green}
    {\bf Name} & {\bf Type} & {\bf Direction} & {\bf Width} & {\bf Description} \\
    \hline \hline 
    rst & digital & in & 1 & System reset\\
    \hline
    en & digital & in & 1 & System enable \\
    \hline
    clk &  digital & in & 1 & 32 MHz system clock \\
    \hline
    sel &  digital & in & 1 & CPU interface \\
    \hline
    write &  digital & in & 1 & CPU interface \\
    \hline
    address &  digital & in & 5 & CPU interface \\
    \hline
    data\_in &  digital & in & 32 & CPU interface \\
    \hline
    ready &  digital & out & 1 & CPU interface \\
    \hline
    FCW &  digital & in &  26 & Channel frequency in MHz (Q12.14) \\
    \hline
    adpll\_mode & digital & in & 2 & ADPLL operation mode \\
    \hline
    data\_mod & digital & in & 1 & Data bit to modulated in TX mode\\
    \hline
    channel\_lock & digital & out & 1 & Channel frequency locking flag \\
    \hline
    DVDD & analog &  & 1 & Digital core supply voltage \\
    \hline
    VDD\_TDC & analog &   & 1 & TDC supply voltage \\
    \hline
    VDD\_DCO & analog &  & 1 & DCO supply voltage \\
    \hline
    DGND & analog &    & 1 & Digital ground \\
    \hline
    GND & analog &    & 1 & Analog ground \\
    \hline
    IREF & analog &    & 1 & Analog reference current (5$\mu$A) \\
    \hline
    CKV\_P & analog &    & 1 & DCO positive output \\
    \hline
    CKV\_N & analog &    & 1 & DCO negative output \\
    \hline
    
    \end{tabular}
  \caption{Interface signals}
  \label{tab:adpll_interface}
\end{table}


\subsection{Memory mapped registers}

The list of memory mapped registers is shown in Table~\ref{tab:adpll_registers}.

\begin{table}[!h]
  \centering
    \begin{tabular}{|p{3.7cm}|p{1cm}|p{1.2cm}|p{0.8cm}|p{7cm}|p{1.2cm}|}
      \hline
      \rowcolor{tab-green}
    {\bf Name} & {\bf Read/ Write} & {\bf Address} & {\bf Width} & {\bf Description} & {\bf Default} after rst \\
    \hline \hline 
    ALPHA\_L     & write & 0 & 4 & Proportional gain in C\_L state & 14\\
    \hline
    ALPHA\_M     & write & 1 & 4 & Proportional gain in C\_M state & 8\\
    \hline
    ALPHA\_S\_RX & write & 2 & 4 & Proportional gain in C\_S state and RX mode & 7\\
    \hline
    ALPHA\_S\_TX & write & 3 & 4 & Proportional gain in C\_L state and TX mode & 4\\
    \hline
    BETA         & write & 4 & 4 & Integral gain in RX mode  & 0\\
    \hline
    LAMBDA\_RX   & write & 5 & 3 & 1st order IIR LPF coeficient in RX mode & 2\\
    \hline
    LAMBDA\_TX   & write & 6 & 3 & 1st order IIR LPF coeficient in TX mode & 2\\
    \hline
    IIR\_N\_RX   & write & 7 & 2 & Number of 1st order IIR LPFs in RX mode & 3\\
    \hline
    IIR\_N\_TX   & write & 8 & 2 & Number of 1st order IIR LPFs in TX mode & 2\\
    \hline
    FCW\_mod     & write & 9 & 5 & FSK modulation freq deviation (lsb=32kHz) & 9\\
    \hline
    DCO\_C\_L\_WORD\_TEST & write & 10 & 5 & DCO Large Cap bank control signed word  & 0\\
    \hline
    DCO\_C\_M\_WORD\_TEST & write & 11 & 8 & DCO Medium Cap bank control signed word  & 0\\
    \hline
    DCO\_C\_S\_WORD\_TEST & write & 12 & 8 & DCO Small Cap bank control signed word  & 0\\
    \hline
    DCO\_PD\_TEST & write & 13 & 1 & DCO power down in TEST mode  & 1 \\
    \hline
    TDC\_PD\_TEST & write & 14 & 1 & TDC power down in TEST mode  & 1 \\
    \hline
    TDC\_PD\_INJ\_TEST & write & 15 & 1 & TDC injection-locking power down in TEST mode  & 1 \\
    \hline
    TDC\_CTR\_FREQ & write & 16 & 3 & TDC ring oscillator control frequency  & 4 \\
    \hline
    DCO\_OSC\_GAIN & write & 17 & 3 & DCO oscillator control gm  & 2 \\
    \hline
    ADPLL\_SOFT\_RST & write & 18 & 3 & ADPLL soft reset  & 0 \\
    \hline
    
    
    \end{tabular}
  \caption{List of meory mapped registers}
  \label{tab:adpll_registers}
\end{table}


\subsection{Operation}

The ADPLL performs two different tasks: RF channel frequency synthesis, and FSK
modulation. While the former is required on both modes (TX and RX), the latter
is only used in the TX mode. Table~\ref{tab:adpll_modes} shows all different
operation modes that can be programmed according to the adpll\_mode[1:0] input.

\begin{table}[!h]
  \centering
    \begin{tabular}{|p{2.5cm}|p{1.5cm}|p{9cm}|}
      \hline
      \rowcolor{tab-green}
    {\bf adpll\_mode[1:0]} & {\bf Mode} & {\bf Description} \\
    \hline \hline 
    0    & PD   & Power down\\
    \hline
    1    & TEST & Test mode where TDC and DCO blocks can be controlled via CPU\\
    \hline
    2    & RX   & Synthetizes FCW using the ADPLL receiver filtering response \\
    \hline
    3    & TX   & Synthetizes FCW using the ADPLL transmitter filtering response, after the channel lock flag goes high, the input data\_mod bit is synthetized at 1 MHz rate\\
    \hline
    
    \end{tabular}
  \caption{Operation modes}
  \label{tab:adpll_modes}
\end{table}




\begin{figure}[!h]
  
  \centering
      {\includegraphics[width=15cm]{./drawings/adpll_flowchart.png}}
  \caption{ADPLL state flowchart}
  \label{fig:adpll_flowchart}
\end{figure}

The operation of the adpll\_ctr module is shown in
Fig. \ref{fig:adpll_flowchart}, by means of its state flowchart state diagram.
After the initial system reset and enable activation, the controller is in the
IDLE state, remaining here if the ADPLL operation mode is PD or TEST. Otherwise,
it goes to the PU state in the next CLK cycle. In the PU state, the controller
successively powers up the DCO and TDC. After 32 clk cycles (3 $\mu$s), the
controller enters into the C\_L state. This state selects the command word of
the large capacitor bank that generates the closest output frequency to the
target frequency. The transition to C\_M state takes place when an internal
flag (lock\_detect) goes up. The C\_M state operates similarly to the C\_L
state but acting on the medium capacitor bank. Once locked, the controller
goes to the C\_S state. This state gives the finest resolution by acting on the
small capacitor bank during 480 clk cycles (15 $\mu$s). After that, the
channel\_lock output flag is activated meaning that the channel tracking is
completed. In the case of RX mode operation, the controller continuous the
channel frequency tracking, otherwise, it performs data modulation within the
channel (TX mode operation).

Fig. \ref{fig:adpll_filters} shows the block diagram of the filters in the loop.
In the RX mode, a proportional filter and integral filter, as well as
a cascade of first-order LPF IIR filters, can be selected and programmed. The
same filters are available in the TX mode, with the exception of the integral
one.

In the case of TEST mode selection, the TDC and DCO are power-down, ready to be
fully controlled by the CPU. 

Note that whenever the FCW or the adpll\_mode input words are changed, the ADPLL
returns to the IDLE state.




\begin{figure}[!h]
  %\vspace{-.5cm}
  \centering
      {\includegraphics[width=18cm]{./drawings/adpll_filters.png}}
  \caption{ADPLL filters block diagram}
  \label{fig:adpll_filters}
\end{figure}

\subsection{ADPLL Controller}

\subsubsection{Interface signals}

The interface signals of the adpll\_ctr module  are
shown in Table~\ref{tab:adpll_ctr_interface}. 

\begin{table}[!h]
  \centering
    \begin{tabular}{|p{3cm}|p{1.5cm}|p{2cm}|p{1.5cm}|p{7cm}|}
      \hline
      \rowcolor{tab-green}
    {\bf Name} & {\bf Type} & {\bf Direction} & {\bf Width} & {\bf Description} \\
    \hline \hline 
    rst & digital & in & 1 & System reset\\
    \hline
    en & digital & in & 1 & System enable \\
    \hline
    clk &  digital & in & 1 & 32 MHz system clock \\
    \hline
    sel &  digital & in & 1 & CPU interface \\
    \hline
    write &  digital & in & 1 & CPU interface \\
    \hline
    address &  digital & in & 5 & CPU interface \\
    \hline
    data\_in &  digital & in & 32 & CPU interface \\
    \hline
    ready &  digital & out & 1 & CPU interface \\
    \hline
    FCW &  digital & in &  26 & Channel frequency in MHz (Q12.14) \\
    \hline
    adpll\_mode & digital & in & 2 & ADPLL operation mode \\
    \hline
    data\_mod & digital & in & 1 & Data bit to modulated in TX mode\\
    \hline
    channel\_lock & digital & out & 1 & Channel frequency locking flag \\
    \hline
    dco\_pd & digital & out  & 1 & DCO power-down\\
    \hline
    dco\_osc\_gain & digital & out  & 2 & DCO gm control\\
    \hline
    dco\_c\_l\_rall & digital & out  & 5 & DCO large cap bank full row control\\
    \hline
    dco\_c\_l\_row & digital & out  & 5 & DCO large cap bank row control\\
    \hline
    dco\_c\_l\_col & digital & out  & 5 & DCO large cap bank column control\\
    \hline
    dco\_c\_m\_rall & digital & out  & 8 & DCO medium cap bank full row control\\
    \hline
    dco\_c\_m\_row & digital & out  & 8 & DCO medium cap bank row control\\
    \hline
    dco\_c\_m\_col & digital & out  & 8 & DCO medium cap bank column control\\
    \hline
     dco\_c\_s\_rall & digital & out  & 8 & DCO small cap bank full row control\\
    \hline
    dco\_c\_s\_row & digital & out  & 8 & DCO small cap bank row control\\
    \hline
    dco\_c\_s\_col & digital & out  & 8 & DCO small cap bank column control\\
    \hline
    tdc\_pd & digital & out  & 1 & TDC power-down\\
    \hline
    tdc\_pd\_inj & digital & out  & 1 & TDC injection-locking power-down\\
    \hline
    tdc\_ctr\_freq & digital & out  & 3 & TDC ring oscillator frequency control\\
    \hline
    tdc\_ripple\_count & digital & in  & 7 & RF ripple counter output\\
    \hline
    tdc\_phase & digital & in  & 16 & TDC ring oscillator phases\\
    \hline
    \end{tabular}
  \caption{Controller interface}
  \label{tab:adpll_ctr_interface}
\end{table}




\subsection{Time-to-Digital Converter}
\label{sec:tdc}
A TDC plays a major role in digital PLLs since converts the local oscillator output frequency into a digital word that is processed by the PLL unit control. This is a high-resolution TDC suitable for digital PLLs operating in the 2-3 GHz frequency band. The TDC employs a novel circuit topology that allows wide range injection-locking (more than 1 GHz) without compromise the TDC DNL.

\subsubsection{Interface signals}

The TDC interface signals are shown in Table~\ref{tab:tdc_interface}. 
\begin{table}[!h]
  \centering
    \begin{tabular}{|p{3cm}|p{1.5cm}|p{2cm}|p{1.5cm}|p{7cm}|}
      \hline
      \rowcolor{tab-green}
    {\bf Name} & {\bf Type} & {\bf Direction} & {\bf Width} & {\bf Description} \\
    \hline \hline   
    tdc\_pd & digital & in  & 1 & TDC power-down\\
    \hline
    tdc\_pd\_inj & digital & in  & 1 & TDC injection-locking power-down\\
    \hline
    clk &  digital & in & 1 & 32 MHz system clock \\
    \hline
    tdc\_ctr\_freq & digital & in  & 3 & TDC ring oscillator frequency control\\
    \hline
    tdc\_ripple\_count & digital & out  & 7 & RF ripple counter output\\
    \hline
    tdc\_phase & digital & out  & 16 & TDC ring oscillator phases\\
    \hline
    DVDD & analog &  & 1 & Digital core supply voltage \\
    \hline
    VDD & analog &   & 1 & TDC supply voltage \\
    \hline
    DGND & analog &    & 1 & Digital ground \\
    \hline
    GND & analog &    & 1 & Analog ground \\
    \hline
    IREF & analog &    & 1 & Analog reference current (5$\mu$A) \\
    \hline
    INJ\_P & analog &    & 1 & DCO positive output injector\\
    \hline
    INJ\_N & analog &    & 1 & DCO negative output injector \\
    \hline
  
    \end{tabular}
  \caption{TDC interface signals}
  \label{tab:tdc_interface}
\end{table}

The TDC employs a ring oscillator that requires an initial settling time at the time the TDC is powered up. At the worst PVT case, simulation results show an RO initial settling time of about 300 ns. Thus, it is recommended to wait for at least 300 ns before the injection of an external RF input signal. After the signal injection activation , it is recommended to wait for at least 100 ns plus five CLK cycles in order to get a valid tdc output word.



\subsection{Digital-Controlled Oscillator}
\label{sec:dco}
\subsubsection{Interface signals}
The DCO interface signals are shown in Table~\ref{tab:dco_interface}. 

\begin{table}[!h]
  \centering
    \begin{tabular}{|p{3cm}|p{1.5cm}|p{2cm}|p{1.5cm}|p{7cm}|}
      \hline
      \rowcolor{tab-green}
    {\bf Name} & {\bf Type} & {\bf Direction} & {\bf Width} & {\bf Description} \\
    \hline \hline    
    pd & digital & in  & 1 & DCO power-down\\
    \hline
    dco\_osc\_gain & digital & in  & 2 & DCO gm control\\
    \hline
    dco\_c\_l\_rall & digital & in  & 5 & DCO large cap bank full row control\\
    \hline
    dco\_c\_l\_row & digital & in  & 5 & DCO large cap bank row control\\
    \hline
    dco\_c\_l\_col & digital & in  & 5 & DCO large cap bank column control\\
    \hline
    dco\_c\_m\_rall & digital & in  & 8 & DCO medium cap bank full row control\\
    \hline
    dco\_c\_m\_row & digital & in  & 8 & DCO medium cap bank row control\\
    \hline
    dco\_c\_m\_col & digital & in  & 8 & DCO medium cap bank column control\\
    \hline
    dco\_c\_s\_rall & digital & in  & 8 & DCO small cap bank full row control\\
    \hline
    dco\_c\_s\_row & digital & in  & 8 & DCO small cap bank row control\\
    \hline
    dco\_c\_s\_col & digital & in  & 8 & DCO small cap bank column control\\
    \hline
    VDD & analog &  & 1 & DCO supply voltage \\
    \hline
    GND & analog &    & 1 & Analog ground \\
    \hline
    IREF & analog &    & 1 & Analog reference current (5$\mu$A) \\
    \hline
    CKV\_P & analog &    & 1 & DCO positive output \\
    \hline
    CKV\_N & analog &    & 1 & DCO negative output \\
    \hline

  
    \end{tabular}
  \caption{DCO interface signals}
  \label{tab:dco_interface}
\end{table}

\newpage 

